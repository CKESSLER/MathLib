\exo \textbf{A propos du dioxyde de soufre}

\vspace{0.3cm}

\textbf{Document $1$}

\vspace{0.3cm}

Le dioxyde de soufre de formule \chemform{SO_{2}} est un polluant atmosphérique. En particulier les moteurs des véhicules à essence réalisent la combustion du soufre présent dans l'essence.
De nombreuses réactions peuvent conduire à la formation du dioxyde de soufre. L'une des plus
courantes est la combustion du sulfure d'hydrogène \chemform{H_{2}S} qui produit également de l'eau.
Une fois dans l'atmosphère, le dioxyde de soufre réagit avec l'eau pour former l'acide sulfurique de formule \chemform{H_{2}SO_{4}}. Cette molécule se dissout dans l'eau pour former des ions sulfates \chemform{SO_{4}^{2-}} et des ions \chemform{H^{+}}.\newline
\textbf{L'augmentation de la concentration en ions \chemform{H^{+}} dans l'eau provoque un abaissement du \chemform{pH} et la formation de pluies acides.}

\vspace{0.3cm}

\textbf{Document $2$ : réactions acide/base}

\vspace{0.3cm}

Les ions \chemform{HO^{-}} peuvent réagir avec les ions \chemform{H^{+}} pour former de l'eau. La réaction est totale et libère de l'énergie sous forme de chaleur. On utilise souvent cette réaction pour déterminer la concentration en ions H+ dans une solution: au moment où les réactifs ont été introduits dans les proportions stœchiométriques, il n'y a plus d'ions \chemform{H^{+}} ni d'ions \chemform{HO^{-}} dans la solution, ce qui permet de déterminer la concentration recherchée.

%\vspace{0.3cm}

\newpage

\textbf{Document $3$ : le BBT, un indicateur coloré}

\vspace{0.3cm}

Le BBT est indicateur coloré acido-basique. Il colore une solution en jaune si la solution est acide, il la colore en bleu si elle est basique. Lorsque la solution est neutre, il donne une couleur intermédiaire verte. Dans ce dernier cas, la concentration en ions \chemform{H^{+}} est de $1,0 . 10^{-7}\text{ }\mole.\liter^{-1}$.

\vspace{0.3cm}

\textbf{Document $4$ : réaction de combustion}

\vspace{0.3cm}

Une combustion est une réaction d'oxydoréduction exothermique au cours de laquelle un combustible réagit avec un comburant. Le comburant le plus courant étant le dioxygène de l'air. Pour se produire, une telle réaction nécessite une énergie d'activation souvent apportée par une
source de chaleur. La chaleur produite par la réaction entretient le mécanisme de combustion.

\vspace{0.3cm}

\textbf{Document $5$ : données de quelques éléments}

%\vspace{0.3cm}

\begin{center}
\begin{tabular}{|c|c|c|c|}
\hline
Elément & Numéro atomique & Masse molaire ($\gram.\mole^{-1}$) & Electronégativité \\
\hline
Hydrogène & $1$ & $1,0$ & $2,2$ \\
\hline
Oxygène & $8$ & $16,0$ & $3,44$ \\
\hline
Soufre & $16$ & $32,1$ & $2,78$ \\
\hline
\end{tabular}
\end{center}

\begin{enumerate}
\item Déterminer la formule de Lewis du sulfure d'hydrogène.
\item En déduire la géométrie de la molécule.
\item Cette molécule est-elle polaire ?
\item Cette molécule est-elle soluble dans l'eau ? Justifier.
\item Proposer une équation pour la réaction de combustion du soufre et pour la combustion du sulfure d'hydrogène.
\item Ecrire l'équation de la dissolution de l'acide sulfurique dans l'eau.
\item Nommer et décrire les $3$ étapes du processus de dissolution.

\vspace{0.3cm}

Au laboratoire, on dispose d'une solution d'acide sulfurique à $13\text{ }\mole.\liter^{-1}$.

\item Déterminer la concentration en ions \chemform{H^{+}} et en ions \chemform{SO_{4}^{2-}} dans la solution.

\vspace{0.3cm}

On prélève $10\text{ }\milli\liter$ de cette solution que l'on introduit dans une fiole jaugée de $1\text{ }\liter$, puis on complète avec de l'eau distillée.

\item Comment s'appelle cette opération ?

\item Décrire les étapes de cette préparation par une série de schémas légendés.

\item Déterminer la nouvelle concentration en ions \chemform{H^{+}} et \chemform{SO_{4}^{2-}}.

\vspace{0.3cm}

Pour vérifier l'affirmation selon laquelle le dioxyde de soufre se dissous dans l'eau et abaisse son \chemform{pH}, on réalise la combustion du soufre et l'on récupère le gaz formé pour le faire barboter dans $50\text{ }\milli\liter$ d'eau distillée. On suppose que la totalité du gaz formé se dissout dans l'eau.\newline
On verse alors de la soude (\chemform{Na^{+}+ HO^{-}}) dans le bécher après y avoir ajouté quelques gouttes de BBT. Il faut verser $15\text{ }\milli\liter$ de soude de concentration $C = 2,0 . 10^{-1}\text{ }\mole.\liter^{-1}$ pour voir la solution virer au vert.

\item En vous appuyant sur les documents, montrer que l'affirmation en gras du document $1$ est vérifiée et déterminer la concentration en ions \chemform{H^{+}} dans la solution.

\end{enumerate}

%\vspace{0.3cm}

\newpage