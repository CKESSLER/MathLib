\bigskip
%%%%%%%%%%%%%%%%%%%%%%%%%%%%%%%%%%%%%%%%%%%%%%%%%%%%%%%%%%%%%%%%%%%%%
\textbf{\textcolor{red}{$\boxed{\text{Corrigé exercice n°1 (*)}}$}}
%%%%%%%%%%%%%%%%%%%%%%%%%%%%%%%%%%%%%%%%%%%%%%%%%%%%%%%%%%%%%%%%%%%%%
\bigskip

\textcolor{red}{Posons la propriété à démontrer : soit $P(n)\text{ : }"u_{n}=3-2^{n}"$.}

\bigskip

\textcolor{red}{\underline{Initialisation} :}

\bigskip

\textcolor{red}{$u_{0}=2$ et $3-2^{0}=3-1=2$, on a donc bien $u_{0}=3-2^{0}$. $P(0)$ est vraie.}

\bigskip

\textcolor{red}{\underline{Hérédité} : soit $n \in \N$, supposons que $P(n)$ est vraie. Montrons que $P(n+1)$ est vraie.}

\bigskip

\textcolor{red}{$u_{n+1}=2u_{n}-3=2\left(3-2^{n}\right)-3=6-2^{n+1}-3=3-2^{n+}$. Donc $P(n+1)$ est vraie.}

\bigskip

\textcolor{red}{\textbf{Conclusion :} $\forall n \in \N\text{, }P(n)$ est vraie, c'est-à-dire :}

\textcolor{red}{$$\boxed{\forall n \in \N\text{, }u_{n}=3-2^{n}}$$}

\bigskip